\documentclass[a4paper]{article}

\newcommand{\Rfunction}[1]{{\texttt{#1}}}
\newcommand{\Robject}[1]{{\texttt{#1}}}
\newcommand{\Rpackage}[1]{{\textit{#1}}}
\newcommand{\Rclass}[1]{{\textit{#1}}}
\newcommand{\Rmethod}[1]{{\textit{#1}}}

\author{HC Ji\footnote{ji.hongchao@csu.edu.cn}}

\usepackage{Sweave}
\begin{document}
\input{KPIC2-concordance}
\setkeys{Gin}{width=1\textwidth} 
\title{Using the KPIC package}
\maketitle

\section{Overview} 
KPIC2 is an effective platform for LC-MS based metabolomics using pure ion chromatograms, which is developed for metabolomics studies. KPIC2 can detect pure ions accurately, align PICs across samples, group PICs to annotate isotope and adduct PICs, fill missing peaks and pattern recognition. High-resolution mass spectrometers like TOF and Orbitrap are more suitable.

\section{Raw Data File Preparation}
The \Rpackage{KPIC} package readspackage reads full-scan LC/MS data from mzXML and mzData files.
\begin{Schunk}
\begin{Sinput}
> library(KPIC)
\end{Sinput}
\end{Schunk}
In the user guide, faahKO dataset is taken as example, though the data is not high-resolution. The files are obtained from the \Rpackage{faahKO}
\begin{Schunk}
\begin{Sinput}
> library(faahKO)
> path <- system.file("CDF", package = "faahKO")
\end{Sinput}
\end{Schunk}

\section{Extraction of Pure Ion Chromatograms}
Extract pure ion chromatograms via optimized K-means clustering of ions in region of interest, and detect peaks of PICs
\begin{Schunk}
\begin{Sinput}
> st <- KPICset(path,range=20,level=5000,alpha=1,itol=0.5,min_snr=3,min_ridge=3,fst=0.4,eval=FALSE) 
\end{Sinput}
\end{Schunk}

\section{Alignment}
Align PICs across samples.
\begin{Schunk}
\begin{Sinput}
> st <- PIAlign(st)
\end{Sinput}
\begin{Soutput}
3 %  is done 
5 %  is done 
8 %  is done 
10 %  is done 
13 %  is done 
15 %  is done 
18 %  is done 
20 %  is done 
23 %  is done 
25 %  is done 
28 %  is done 
30 %  is done 
33 %  is done 
35 %  is done 
38 %  is done 
40 %  is done 
43 %  is done 
45 %  is done 
48 %  is done 
50 %  is done 
53 %  is done 
55 %  is done 
58 %  is done 
60 %  is done 
63 %  is done 
65 %  is done 
68 %  is done 
70 %  is done 
73 %  is done 
75 %  is done 
78 %  is done 
80 %  is done 
83 %  is done 
85 %  is done 
88 %  is done 
90 %  is done 
93 %  is done 
95 %  is done 
98 %  is done 
\end{Soutput}
\end{Schunk}

\section{Grouping}
Grouping the features. Including grouping the isotopic ions and adducts with the main features and grouping main features across samples.
\begin{Schunk}
\begin{Sinput}
> gp <- PICgroup(st)
\end{Sinput}
\end{Schunk}

\section{Generate the peak matrix}
Summarize extracted information into a data matrix.
\begin{Schunk}
\begin{Sinput}
> ma <- getDataMatrix(gp,std='maxo')
\end{Sinput}
\end{Schunk}

\section{Filling missing peaks}
Fill missing peaks based on EIBPC.
\begin{Schunk}
\begin{Sinput}
> fp <- fillPeaks(ma)