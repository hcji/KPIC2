\documentclass[a4paper]{article}

\newcommand{\Rfunction}[1]{{\texttt{#1}}}
\newcommand{\Robject}[1]{{\texttt{#1}}}
\newcommand{\Rpackage}[1]{{\textit{#1}}}
\newcommand{\Rclass}[1]{{\textit{#1}}}
\newcommand{\Rmethod}[1]{{\textit{#1}}}

\author{HC Ji\footnote{ji.hongchao@csu.edu.cn}}

\usepackage{Sweave}
\begin{document}
\input{KPIC2-concordance}
\setkeys{Gin}{width=1\textwidth} 
\title{Using the KPIC package}
\maketitle

\section{Overview} 
KPIC2 is an effective platform for LC-MS based metabolomics using pure ion chromatograms, which is developed for metabolomics studies. KPIC2 can detect pure ions accurately, align PICs across samples, group PICs to annotate isotope and adduct PICs, fill missing peaks and pattern recognition. High-resolution mass spectrometers like TOF and Orbitrap are more suitable.

\section{Raw Data File Preparation}
The \Rpackage{KPIC} package reads full-scan LC/MS data from mzXML and mzData files.
\begin{Schunk}
\begin{Sinput}
> library(KPIC)
\end{Sinput}
\end{Schunk}
In the user guide, an example dataset is taken as example, which is different concentraction of standard compounds solution mixed with plasma extraction.
\begin{Schunk}
\begin{Sinput}
> library(faahKO)
> path <- 'E:/LC-MS data/example'
\end{Sinput}
\end{Schunk}

\section{Extraction of Pure Ion Chromatograms}
Extract pure ion chromatograms via optimized K-means clustering of ions in region of interest, and detect peaks of PICs
\begin{Schunk}
\begin{Sinput}
> st <- KPICset(path,roi_range=0.2,level=70000) 